% Options for packages loaded elsewhere
\PassOptionsToPackage{unicode}{hyperref}
\PassOptionsToPackage{hyphens}{url}
\documentclass[
]{article}
\usepackage{xcolor}
\usepackage{amsmath,amssymb}
\setcounter{secnumdepth}{-\maxdimen} % remove section numbering
\usepackage{iftex}
\ifPDFTeX
  \usepackage[T1]{fontenc}
  \usepackage[utf8]{inputenc}
  \usepackage{textcomp} % provide euro and other symbols
\else % if luatex or xetex
  \usepackage{unicode-math} % this also loads fontspec
  \defaultfontfeatures{Scale=MatchLowercase}
  \defaultfontfeatures[\rmfamily]{Ligatures=TeX,Scale=1}
\fi
\usepackage{lmodern}
\ifPDFTeX\else
  % xetex/luatex font selection
\fi
% Use upquote if available, for straight quotes in verbatim environments
\IfFileExists{upquote.sty}{\usepackage{upquote}}{}
\IfFileExists{microtype.sty}{% use microtype if available
  \usepackage[]{microtype}
  \UseMicrotypeSet[protrusion]{basicmath} % disable protrusion for tt fonts
}{}
\makeatletter
\@ifundefined{KOMAClassName}{% if non-KOMA class
  \IfFileExists{parskip.sty}{%
    \usepackage{parskip}
  }{% else
    \setlength{\parindent}{0pt}
    \setlength{\parskip}{6pt plus 2pt minus 1pt}}
}{% if KOMA class
  \KOMAoptions{parskip=half}}
\makeatother
\setlength{\emergencystretch}{3em} % prevent overfull lines
\providecommand{\tightlist}{%
  \setlength{\itemsep}{0pt}\setlength{\parskip}{0pt}}
\usepackage{bookmark}
\IfFileExists{xurl.sty}{\usepackage{xurl}}{} % add URL line breaks if available
\urlstyle{same}
\hypersetup{
  pdftitle={HW 3 v1},
  hidelinks,
  pdfcreator={LaTeX via pandoc}}

\title{HW 3 v1}
\author{}
\date{}

\begin{document}
\maketitle

\subsection{The Assignment}\label{the-assignment}

\subsubsection{Overview}\label{overview}

Up until now, you have built isolated components: a Chat service, a
Ticketing service, or an AI service. In this final phase, you are not
building new features from scratch. Instead, you are acting as a
\textbf{Systems Integrator}.

Your goal is to assemble these components into a single, cohesive
pipeline that allows a user in a chat interface to manage work tickets
using natural language.

The final product must be able to:

\begin{itemize}
\tightlist
\item
  \textbf{Answer questions} by leveraging an AI model.
\item
  \textbf{Fetch and summarize tickets} from a ticketing system (e.g.,
  "Show me my 3 most recent open tickets").
\item
  \textbf{Take action on tickets} (e.g., "Close this ticket").
\end{itemize}

\subsubsection{User Flow}\label{user-flow}

Here is the specific workflow your deployed application must support:

\begin{enumerate}
\tightlist
\item
  \textbf{Input:} User types a command in Chat (e.g., \emph{"Create a
  ticket for fixing the login bug"}).
\item
  \textbf{Routing:} The Chat application sends this text to the AI
  Service.
\item
  \textbf{Reasoning:} The AI Service analyzes the text, determines the
  intent (\texttt{create\_ticket}), extracts the data
  (\texttt{title="Fix\ login\ bug"}), and returns a structured tool
  call.
\item
  \textbf{Execution:} The application executes this call against the
  standardized Ticket Interface.
\item
  \textbf{Response:} The Ticket Service confirms the action, and the
  Chat Interface relays the success message back to the user.
\end{enumerate}

\subsubsection{Instructions}\label{instructions}

The core steps are:

\begin{enumerate}
\tightlist
\item
  \textbf{Integration:} Your primary task is to make three disparate
  systems---Chat, Tickets, and AI---communicate effectively.
\item
  \textbf{Deployment:} This is not a local-only project. You will deploy
  your application and manage its infrastructure as code.
\item
  \textbf{Observability:} Your deployed application must emit telemetry
  data to monitor its health and performance.
\end{enumerate}

\paragraph{Interface}\label{interface}

\href{https://github.com/ivanearisty/OSS-APIs}{Repository}

\paragraph{IoC \& Telemetry}\label{ioc-telemetry}

Your application must be deployed, and its infrastructure must be
managed using an \textbf{Infrastructure as Code (IaC)} tool like
Terraform or AWS CloudFormation. You are responsible for provisioning
the necessary resources (e.g., servers, databases, environment
variables) in a repeatable and automated way.

Furthermore, your deployed application must emit \textbf{telemetry
data}. This is non-negotiable and critical for understanding the
performance of a live service. You must implement monitoring for:

\begin{itemize}
\tightlist
\item
  \textbf{Request Latency:} How long does each API call or user
  interaction take?
\item
  \textbf{Success Rate:} What percentage of requests are completed
  successfully?
\item
  \textbf{Failure Rate:} What percentage of requests result in errors?
\end{itemize}

You should use a monitoring or observability platform to collect and
visualize this data.

I recommend that you have a working version of this by the HW 3 Amended
\textgreater{} Second Submission deadline

\paragraph{First submission}\label{first-submission}

By the turn of the month, your individual team\textquotesingle s project
must be refactored to implement the shared, standardized API for your
vertical. This is a hard deadline, as other teams will be depending on
this standardized interface for the next phase.

\paragraph{Second Submission}\label{second-submission}

Choose \textbf{one} of the other two verticals (either Tickets or AI)
and integrate their system into your project. For example, if you are a
Chat team, you might integrate an AI system first. Your submission must:

\begin{itemize}
\tightlist
\item
  Demonstrate successful integration with at least two different
  providers from that vertical (e.g., a Chat bot working with both
  OpenAI and Gemini).
\item
  Include integration tests that verify the two systems are working
  together correctly.
\end{itemize}

I recommend you also have your IoC and Telemetry setup by now.

\paragraph{Final Submission}\label{final-submission}

This is the final deliverable. It includes your complete, three-part
system, documentation, and a video demonstration.

\begin{itemize}
\tightlist
\item
  \textbf{Pull Request:} A clean, well-documented PR with your final
  code.
\item
  \textbf{Video Demonstration:} You must record a video that includes:

  \begin{itemize}
  \tightlist
  \item
    A clear explanation, in your own words, of how your complete project
    works.
  \item
    A live demonstration of its functionality, showcasing integration
    with different providers (e.g., swapping Jira for Trello).
  \item
    A walkthrough of your CircleCI deployment pipeline.
  \item
    An explanation of the tests being run on the cloud.
  \item
    A high-level overview of your end-to-end (e2e) and integration tests
    and what they verify.
  \item
    A view of your telemetry dashboard showing request latency and
    success/failure rates.
  \end{itemize}
\end{itemize}

\subsubsection{Deadlines}\label{deadlines}

\begin{itemize}
\tightlist
\item
  \textbf{11/21:} Draft of this assignment is provided.
\item
  \textbf{11/26:} Draft is frozen.
\item
  \textbf{12/27 - Interface Deliverable:} Verticals must submit their
  agreed-upon shared interface memo and individual team alignment plans.
\item
  \textbf{12/8 - First Submission:} All teams must have their code
  aligned with the shared API.
\item
  \textbf{12/12 - Second Submission:} First cross-vertical integration
  is complete.
\item
  \textbf{12/14:} Peer feedback and TA feedback on the first
  integration.
\item
  \textbf{12/19 - Final Submission:} Final project, including the video
  demonstration, is submitted.
\end{itemize}

\end{document}
